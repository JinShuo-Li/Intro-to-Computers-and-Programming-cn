\documentclass[a4paper, 12pt]{article}

% 引入中文支持
\usepackage[UTF8]{ctex}

% 页面边距设置
\usepackage{geometry}
\geometry{left=2.5cm, right=2.5cm, top=2.5cm, bottom=2.5cm}

% 列表宏包,用于自定义编号和间距
\usepackage{enumitem}

% 颜色宏包
\usepackage{xcolor}

% 超链接宏包
\usepackage{hyperref}

% 定义代码样式的命令 (等宽字体 + 深蓝色)
\newcommand{\code}[1]{\texttt{\color{blue!60!black}#1}}
\newcommand{\kw}[1]{\textbf{#1}} % 关键词加粗

% 自定义 001, 002 样式的计数器显示
\newcommand{\threedigits}[1]{%
  \ifnum#1<10 00#1%
  \else\ifnum#1<100 0#1%
  \else #1%
  \fi\fi%
}

\title{\textbf{Python 计算导论 Cheating Sheet}}
\author{}
\date{}

\begin{document}

\maketitle

% 设置正文列表
\begin{enumerate}[label=\textbf{\protect\threedigits{\value*}.}, leftmargin=3.5em, itemsep=0.8ex, parsep=0.5ex]

    \item \code{print()}
    \begin{itemize}[label=$\circ$]
        \item 转义字符: \code{\textbackslash n}: 换行 (new line), \code{\textbackslash t}: 制表符 (tab)。(在字符串前加 \code{r} 可以取消转义字符: \code{r"..."})
        \item \code{""}, \code{""""""}: 前者不自动换行,需要用 \code{\textbackslash n} 换行;后者自动换行,需要用 \code{\textbackslash} 取消换行。
        \item f-string: Formatted string literals (格式化字符串字面量)。
    \end{itemize}

    \item \code{ord()}: 转化字符为 ASCII 码。
    
    \item \code{chr()}: 转化 ASCII 码为字符。
    
    \item \kw{运算符}: \code{+} \code{-} \code{*} \code{/}: 加减乘除; \code{//}: 整除; \code{\%}: 求余 (模); \code{**}: 乘方。
    
    \item \code{type(n)}: 返回 n 的类型。
    
    \item \code{id(n)}: 返回 n 的内存地址。
    
    \item \code{int()}, \code{float()}, \code{str()}: 类型转换。
    
    \item \code{eval("...")}: 识别字符串,计算表达式。
    
    \item \code{bool()}: 当且仅当空容器、0、\code{False}、\code{None} 时返回 \code{False},否则返回 \code{True}。
    
    \item \code{range(start=0, end, step=1)}: 输出从 start 到 end 的以 step 为公差的等差数列,\textbf{不包括 end},产物是迭代器。
    
    \item \kw{循环控制}: 
    \begin{itemize}
        \item \code{break}: 循环终止,执行循环后的语句。
        \item \code{continue}: 本次循环结束,继续下一个循环。
        \item \code{return}: 函数终止。
    \end{itemize}
    
    \item \code{return a if expre else b}: 如果 expre 满足,输出 a,否则输出 b。
    
    \item \code{for/while-else}: \code{else} 后的语句执行当且仅当循环正常结束。
    
    \item \kw{位运算}: \code{\&}: and, \code{|}: or, \code{\textasciitilde}: Not, \code{\^{}}: XOR (异或), \code{>>}: 右移, \code{<<}: 左移。
    
    \item \code{global} 关键字: 函数内部变量声明为全局变量,调用并修改外部变量。
    
    \item \code{*args} 和 \code{**kwargs}: 接受任意数量的变量,前者包装为 \code{tuple},后者包装为 \code{dictionary}。注意\code{*}和\code{**}是解包操作。
    
    \item \code{list(a)}: a 如果是 string,那么每个字符为列表的一个元素;a 如果是迭代器,那么自动生成列表。
    
    \item \code{list.index(value, start=0, end=-1)}: 搜索索引,end 不被包括在内。默认全局,返回第一个索引。
    
    \item \code{list.count(x)}: 返回 x 出现的次数。
    
    \item \code{list.clear()}: 删除所有元素; \code{list.pop(index)}: 根据索引删除 (并返回该删除的值)。但是用法是 \code{del list[index]}。
    
    \item \code{list.remove(value)}: 删除某个数值,但是不返回; \code{del item}: 删除 item。
    
    \item \code{list.insert(index, value)}: 在某个位置插入 value,index 是插入后的索引位置。一定注意是插入后的位置。
    
    \item \code{list.extend(list')}: 在 list 后面加入 list',改变 list; \code{list + list'} 返回新的列表。这点和字符串类似。
    
    \item \code{list.sort()}: 同类型可以排序; \code{list.reverse()}: 翻转列表。
    
    \item \code{list[i:j]}: 切片,依旧包含 i 不包含 j,生成新的列表; \code{list[a:]}: 从 a 切到最后; \code{list[:a]}: 从开始切到 a-1。
    
    \item \code{list[i:j:k]}: i 开始,j-1 结束,k 是步长。切片一定不会超限。
    
    \item \code{list[:]}: 复制列表; \code{list.copy()}: 浅复制,只复制一层; \code{copy.deepcopy(list)}: 全部复制 (需 \code{import copy})。
    
    \item \kw{列表推导}: \code{[expression for target in iterable if condition]}。
    
    \item \code{for x in list}: 无法获取 index; \code{for i in range(len(list))}: 无法直接获取值。用\code{enumerate}更好。
    
    \item \code{for i, j in enumerate(list)}: i 是索引,j 是索引对应的值。
    
    \item \code{tuple}: 是不可修改的。但是 tuple 内部的元素倘若是可修改的(如列表),则可以修改该元素的内容。
    
    \item \code{dictionary = \{key\_1: value\_1, ...\}}: 字典定义。
    
    \item \kw{Key的限制}: 字典的关键字必须是不可改变的 (immutable)。\code{list}, \code{dictionary} 不可以做 key,但是 \code{tuple} 可以。
    
    \item \kw{字典访问}: \code{dict[key]}; \code{dict[key] = new\_value}: 修改或者重新赋值 (取决于 key 是否存在)。
    
    \item \code{get(key, default)}: 查 key,存在返回对应值,否则返回 default 值。不存在\textbf{不插入} \code{\{key: default\}}。
    
    \item \code{setdefault(key, default)}: 除了会在不存在的时候插入这对值,别的都和 \code{get} 一样。显然用get方法更安全。
    
    \item \code{dict.popitem()}: 删除最后一个插入的 key 和 value。
    
    \item \code{dictionary}: 是可修改的,可以复制,类似 list。
    
    \item \kw{字典融合}: \code{z = \{**x, **y\}}。如果 key 重复,后者覆盖前者。
    
    \item \code{dict.keys()}: 返回所有的键为一个 iterable,需自行转 list 或 tuple。
    
    \item \code{dict.values()}: 返回所有的值为一个 iterable。
    
    \item \code{dict.items()}: 返回所有的 (key, value) 对为一个 iterable。
    
    \item \code{str.split(mark=" ")}: 根据 mark 分裂字符串成列表,默认为空格。
    
    \item \code{str.count()}: string 也具有 count 的功能,可以在一定范围内寻找。
    
    \item \code{str.find(sub, start, end)}: 在范围内找 sub 的第一个 index,end 不被包括在内。返回 -1 意味着不存在。
    
    \item \code{import module}: 调用 module 的函数的时候需要用 \code{module.func()}。
    
    \item \code{from module import func}: 从顶层开始搜索,直到文件夹为止。
    
    \item \kw{时间复杂度 (List/Tuple/Str)}: 查找、插入、删除: $O(n)$; 访问、末尾加入: $O(1)$。但是插入在开头是 $O(n)$。
    
    \item \kw{时间复杂度 (Dict/Set)}: 运用哈希表,空间复杂度高,但是上述操作对字典和集合都是 $O(1)$。
    
    \item \kw{错误类型 (Exceptions)}:
    \begin{description}[style=multiline, leftmargin=4.8cm]
        \item[\code{SyntaxError}] 语法错误,比如括号不封闭。
        \item[\code{IndentationError}] 缩进错误,该对齐的没对齐。
        \item[\code{NameError}] 变量没找到,大概率是没定义。
        \item[\code{AttributeError}] 属性错误,大概率类型搞错了或对象没有该属性。
        \item[\code{TypeError}] 类型错误,瞎操作导致的。
        \item[\code{ValueError}] 数值错误,值不在容器中或参数无效。
        \item[\code{IndexError}] 索引错误,索引超限。
        \item[\code{KeyError}] 键值错误,字典中不存在该键。
        \item[\code{ZeroDivisionError}] 数学错误,以 0 为除数。
    \end{description}
    
    \item \code{random.randint(a, b)}: 返回一个随机整数 $N$,满足 $a \le N \le b$。
    
    \item \code{random.randrange(start, stop, step)}: 返回一个整数,但是 stop 不包括在范围内。
    
    \item \code{random.random()}: 返回一个 0 到 1 之间的随机浮点数。
    
    \item \code{random.uniform(a, b)}: 返回一个 a, b 之间的随机浮点数。
    
    \item \code{random.seed(seed=sys\_time)}: 根据种子生成一个随机浮点数。不指定默认是系统时间。
    
    \item \code{random.choice(iterables)}: 从容器中随机选一个出来。
    
    \item \code{random.choices(iterables, weights, cum\_weights, size)}: 从容器中随机一个 size 的子列出来。
    
    \item \code{random.shuffle(iter)}: 打乱; \code{random.sample(iter, k)}: 随机选长度为 k 的子容器出来。
    
    \item \kw{sys 模块}:
    \begin{itemize}
        \item \code{sys.version}: Python 版本。
        \item \code{sys.argv}: 获取传给 Python 的命令行参数。 \code{argv[0]}: 文件名, \code{argv[1]}: 第一个参数...
        \item \code{sys.path}: Python 搜索的路径列表。
        \item \code{sys.exit([arg])}: 退出程序。
        \item \code{sys.maxsize}: 变量能容纳的最大整数。
        \item \code{sys.stdin}: 标准输入流; \code{sys.stdout}: 标准输出流; \code{sys.stderr}: 标准错误流。
    \end{itemize}
    
    \item \kw{os 模块}:
    \begin{itemize}
        \item \code{os.name}: 返回操作系统名称 (nt: Windows, posix: Linux)。
        \item \code{os.environ}: 读取环境变量。
        \item \code{os.chdir()}: 改变工作目录,等价于在命令行执行 cd 命令。
        \item \code{os.mkdir()}: 创建单个目录,只能创建一级,存在会报错。
        \item \code{os.rename()}: 重命名。若已经存在,在 win 环境下会失败,Linux 会覆盖。
        \item \code{os.remove()}: 删除文件 (删除文件夹用 \code{os.rmdir()})。
        \item \code{os.startfile()}: 启动程序运行,启动关联的默认程序。
        \item \code{os.walk()}: 遍历目录树,产生三元组 \code{(dirpath, dirnames, filenames)}。
    \end{itemize}
    
    \item \kw{open()}: \code{obj = open(file\_name, mode, buffering)} (名字, 模式, 缓冲策略)
    \begin{itemize}
        \item \code{r}: 只读,文件必须存在否则报错。
        \item \code{w}: 只写,会覆盖文件,不存在则创建。
        \item \code{a}: 在文件末尾追加,不存在则创建。
        \item \code{x}: 独占创建,必须不存在,否则报错。
        \item \code{*b}: 加在 rwax 后面,二进制模式,用来处理非文本文件。
        \item \code{*t}: 文本模式,默认模式,通常省略。
        \item \code{*+}: 可读写。
    \end{itemize}
    
    \item \code{obj.closed}: 判断文件是否关闭; \code{obj.name}: 返回文件名; \code{obj.mode}: 文件读写模式。
    
    \item \code{obj.write(string)}: 写入字符串到文件中,不会自动换行。 \code{obj.writelines(list)}: 注意没有 \code{\textbackslash n} 不会自动换行。
    
    \item \code{read(n=None)}: 读取 n 个字节,返回一个字符串,默认全部返回。
    
    \item \code{readline()}: 逐行读取,只读取一行,保留 \code{\textbackslash n},返回字符串。
    
    \item \code{readlines()}: 一次性读取所有行,返回每一行字符串组成的列表,保留 \code{\textbackslash n}。
    
    \item \code{with open(name, mode) as obj}: 缩进块内操作,运行结束自动关闭,安全。
    
    \item \code{for line in obj}: file 对象是可迭代的。
    
    \item \code{iter(obj)}: obj 可以为元组、列表等可迭代的东西,这个函数可以获取它的迭代器。
    
    \item \code{next(iter)}: 对于一个迭代器,每次读下一个迭代的内容。
    
    \item \kw{itertools 模块}:
    \begin{itemize}
        \item \code{product(p, q, ...)}: 返回笛卡尔积。
        \item \code{permutations(p, r)}: 返回一个 r 长度的所有可能排序,无重复元素的元组。
        \item \code{combinations(p, r)}: 返回一个 r 长度的有序的、无重复元素的元组。
    \end{itemize}
    
    \item \code{callable()}: 对于一个函数 \code{test = func()},\code{callable(test)} 返回 True 当且仅当 func 可以被调用。
    
    \item \code{@classmethod}: 标记静态方法 (类方法)。
    
    \item \code{@decorator\_name}: 装饰器,输入函数为变量的方法。
    
    \item \code{@property}: 性质,用于标记方法为属性。
    
    \item \kw{海象运算符}: 允许开发者在表达式内部进行变量赋值: \code{x := 1 + 1}。
    
    \item \kw{Lambda 匿名函数}: \code{lambda var\_1, var\_2: expression},用于在表达式内部处理。
    
    \item \code{sorted(iterable, key=func)}: sorted 高阶用法,通过比较 func(item) 的值来排序。
    
    \item \kw{List 高阶用法}: lambda 函数也可以是一个字典的 value。
    
    \item \code{map(func, iterable)}: 对 iterable 的每个元素使用 func,输出同款 iterable (在 Python 3 中是迭代器)。
    
    \item \code{filter(func, iterable)}: 遍历 iterable 中的每个元素,若 \code{func(i) == True},输出到新容器中。
    
    \item \code{nonlocal}: \code{global} 是直接打通到最外层,\code{nonlocal} 只打通到上一层嵌套函数。注意只有在修改的时候才需要这俩。
    
    \item \kw{LEGB}: 查询顺序: 局部作用域 (Local) $\to$ 外层函数作用域 (Enclosing) $\to$ 全局作用域 (Global) $\to$ 内置作用域 (Built-in)。

\end{enumerate}

\end{document}